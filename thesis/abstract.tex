\section*{Abstract}
\pagenumbering{roman}

Homotopy type theory is a relatively new field which results from the surprising blend of algebraic topology (\textit{homotopy}) and type theory (\textit{type}) that tries to serve as a theoretical base for theorem proving software.
This setting is particularly apt for synthetic homotopy theory.

In this work, we introduce how the programming language Agda can be used for proof verification, by examining the construction of the fundamental group of the circle $\mathbb{S}^{1}$.
Then, trying to obtain the fundamental group of the real projective plane $\mathbb{R}\mathsf{P}^2$, we end up exploring a new alternative construction and partially proving its correction.

\begin{otherlanguage}{catalan}
\section*{Resum}

La teoria homotòpica de tipus és un camp relativament nou que resulta de la sorprenent combinació de topologia algebraica (\textit{homotòpica}) i teoria de tipus (\textit{tipus}) que intenta servir com una base teòrica per a \textit{software} per demostrar teoremes.
Aquest context és particularment adient per la teoria d'homotopia sintètica.

En aquest treball, introduïm de quina manera es pot utilitzar el llenguatge de programació Agda per a verificar demostracions, examinant la construcció del grup fonamental del cercle $\mathbb{S}^{1}$.
Després, intentant obtenir el grup fonamental del pla projectiu real $\mathbb{R}\mathsf{P}^2$, n'acabem explorant una nova construcció alternativa i demostrem parcialment que és correcta.
\end{otherlanguage}

{\let\thefootnote\relax\footnote{2010 Mathematics Subject Classification. 03B35, 03B70, 18G55}}
